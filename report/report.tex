\title{Babble}
\author{Wander Nauta (s1380893) \and Derk Snijders (s0000000) }
\date{\today}

\documentclass[a4paper]{article}

\usepackage[sc]{mathpazo}
\usepackage{parskip}
\usepackage{microtype}
\usepackage[utf8]{inputenc}
\usepackage{inconsolata}
\usepackage{appendix}
\usepackage{color}
\usepackage{fancyvrb}
\usepackage{titlesec}
\usepackage{nag}

\newcommand{\sectionbreak}{\clearpage}
\input{pygments}

\begin{document}

\begin{titlepage}
\maketitle
\end{titlepage}

\section{Language summary}

Babble is a compiled, duck-typed, object-oriented programming language for the Java Virtual machine.
It has syntax and semantics inspired by the Smalltalk programming language, but the compile-once-run-anywhere development cycle from Java.
It has no primitives, no statements like \texttt{if} or \texttt{while}, and no type annotations.
Instead, most everything in Smalltalk is done using message sends.

In Babble, as in Smalltalk, everything is an object, and objects can be interacted with by sending them messages.
If the receiving object understands the message, it acts upon it, sending messages to other objects if required, and finally sending a reply back to the source of the message.
(The process of sending messages would be described as `calling methods' in other languages, while sending a reply would be called `returning a value'.
We use the terms interchangeably.)

Babble is duck typed.
It is a (run-time) error if a receiving object does not understand a message.

Like Smalltalk, Babble has the concept of blocks: pieces of code that can be passed as values.
Blocks can take arguments, and will also do lambda closure.

\section{Detailed language description}

\section{Description of the software}

\section{Test plan and results}

The Babble compiler itself is tested using unit tests, written in Java, while the runtime system is tested using integration tests, written in Babble.
Both can be executed using Maven by running \texttt{mvn verify}.

\section{Conclusions}

\clearpage

\begin{appendices}

\section{ANTLR grammar}

\begin{Verbatim}[commandchars=\\\{\}]
\PY{k}{grammar}\PY{+w}{ }\PY{n+nc}{Babble}\PY{p}{;}

\PY{n+nl}{program}\PY{+w}{ }\PY{p}{:}\PY{+w}{ }\PY{n+nv}{clazz}\PY{o}{*}\PY{+w}{ }\PY{p}{;}

\PY{n+nl}{clazz}\PY{+w}{ }\PY{p}{:}\PY{+w}{ }\PY{n+nv}{classname}\PY{o}{=}\PY{n+no}{ID}\PY{+w}{ }\PY{o}{(}\PY{n+no}{EXTENDS}\PY{+w}{ }\PY{l+s}{\PYZsq{}:\PYZsq{}}\PY{+w}{ }\PY{n+nv}{superclass}\PY{o}{=}\PY{n+no}{ID}\PY{o}{)}\PY{o}{?}\PY{+w}{ }\PY{l+s}{\PYZsq{}[\PYZsq{}}\PY{+w}{ }\PY{n+nv}{mthd}\PY{o}{*}\PY{+w}{ }\PY{l+s}{\PYZsq{}].\PYZsq{}}\PY{+w}{ }\PY{p}{;}

\PY{n+nl}{mthd}\PY{+w}{ }\PY{p}{:}\PY{+w}{ }\PY{o}{(}\PY{n+no}{ID}\PY{+w}{ }\PY{l+s}{\PYZsq{}:\PYZsq{}}\PY{+w}{ }\PY{n+no}{ID}\PY{o}{)}\PY{o}{+}\PY{+w}{ }\PY{l+s}{\PYZsq{}[\PYZsq{}}\PY{+w}{ }\PY{n+nv}{sequence}\PY{+w}{ }\PY{l+s}{\PYZsq{}].\PYZsq{}}\PY{+w}{   }\PY{o}{\PYZsh{}}\PY{+w}{ }\PY{n+no}{KeywordMethod}
\PY{+w}{     }\PY{o}{|}\PY{+w}{ }\PY{n+no}{ID}\PY{+w}{ }\PY{l+s}{\PYZsq{}[\PYZsq{}}\PY{+w}{ }\PY{n+nv}{sequence}\PY{+w}{ }\PY{l+s}{\PYZsq{}].\PYZsq{}}\PY{+w}{             }\PY{o}{\PYZsh{}}\PY{+w}{ }\PY{n+no}{UnaryMethod}
\PY{+w}{     }\PY{p}{;}

\PY{n+nl}{sequence}\PY{+w}{ }\PY{p}{:}\PY{+w}{ }\PY{o}{(}\PY{n+nv}{expr}\PY{+w}{ }\PY{o}{(}\PY{l+s}{\PYZsq{}.\PYZsq{}}\PY{o}{+}\PY{+w}{ }\PY{n+nv}{expr}\PY{o}{)}\PY{o}{*}\PY{o}{)}\PY{o}{?}\PY{+w}{ }\PY{l+s}{\PYZsq{}.\PYZsq{}}\PY{o}{?}\PY{+w}{ }\PY{p}{;}

\PY{n+nl}{expr}\PY{+w}{ }\PY{p}{:}\PY{+w}{ }\PY{n+no}{ID}\PY{+w}{ }\PY{l+s}{\PYZsq{}:=\PYZsq{}}\PY{+w}{ }\PY{n+nv}{expr}\PY{+w}{                        }\PY{o}{\PYZsh{}}\PY{+w}{ }\PY{n+no}{Assignment}
\PY{+w}{     }\PY{o}{|}\PY{+w}{ }\PY{n+nv}{rcv}\PY{o}{=}\PY{n+nv}{expr}\PY{+w}{ }\PY{n+nv}{method}\PY{o}{=}\PY{n+no}{ID}\PY{+w}{                  }\PY{o}{\PYZsh{}}\PY{+w}{ }\PY{n+no}{UnarySend}
\PY{+w}{     }\PY{o}{|}\PY{+w}{ }\PY{n+nv}{rcv}\PY{o}{=}\PY{n+nv}{expr}\PY{+w}{ }\PY{n+nv}{method}\PY{o}{=}\PY{n+no}{OPERATOR}\PY{+w}{ }\PY{n+nv}{arg}\PY{o}{=}\PY{n+nv}{expr}\PY{+w}{   }\PY{o}{\PYZsh{}}\PY{+w}{ }\PY{n+no}{InfixSend}
\PY{+w}{     }\PY{o}{|}\PY{+w}{ }\PY{n+nv}{rcv}\PY{o}{=}\PY{n+nv}{expr}\PY{+w}{ }\PY{o}{(}\PY{n+no}{ID}\PY{+w}{ }\PY{l+s}{\PYZsq{}:\PYZsq{}}\PY{+w}{ }\PY{n+nv}{subexpr}\PY{o}{)}\PY{o}{+}\PY{+w}{          }\PY{o}{\PYZsh{}}\PY{+w}{ }\PY{n+no}{KeywordSend}
\PY{+w}{     }\PY{o}{|}\PY{+w}{ }\PY{o}{(}\PY{n+no}{ID}\PY{+w}{ }\PY{l+s}{\PYZsq{}:\PYZsq{}}\PY{+w}{ }\PY{n+nv}{subexpr}\PY{o}{)}\PY{o}{+}\PY{+w}{                   }\PY{o}{\PYZsh{}}\PY{+w}{ }\PY{n+no}{GlobalKeywordSend}\PY{+w}{ }\PY{c}{//TODO put in IRtree}
\PY{+w}{     }\PY{o}{|}\PY{+w}{ }\PY{n+nv}{subexpr}\PY{+w}{                             }\PY{o}{\PYZsh{}}\PY{+w}{ }\PY{n+no}{LoneExpr}
\PY{+w}{     }\PY{p}{;}

\PY{n+nl}{subexpr}\PY{+w}{ }\PY{p}{:}\PY{+w}{ }\PY{n+nv}{value}\PY{o}{=}\PY{n+no}{INTEGER}\PY{+w}{                 }\PY{o}{\PYZsh{}}\PY{+w}{ }\PY{n+no}{IntLit}
\PY{+w}{        }\PY{o}{|}\PY{+w}{ }\PY{n+nv}{string}\PY{o}{=}\PY{n+no}{STRING}\PY{+w}{                 }\PY{o}{\PYZsh{}}\PY{+w}{ }\PY{n+no}{StrLit}
\PY{+w}{        }\PY{o}{|}\PY{+w}{ }\PY{n+no}{TRUE}\PY{+w}{                          }\PY{o}{\PYZsh{}}\PY{+w}{ }\PY{n+no}{TrueLit}
\PY{+w}{        }\PY{o}{|}\PY{+w}{ }\PY{n+no}{FALSE}\PY{+w}{                         }\PY{o}{\PYZsh{}}\PY{+w}{ }\PY{n+no}{FalseLit}
\PY{+w}{        }\PY{o}{|}\PY{+w}{ }\PY{n+no}{NIL}\PY{+w}{                           }\PY{o}{\PYZsh{}}\PY{+w}{ }\PY{n+no}{NilLit}
\PY{+w}{        }\PY{o}{|}\PY{+w}{ }\PY{n+no}{ID}\PY{+w}{                            }\PY{o}{\PYZsh{}}\PY{+w}{ }\PY{n+no}{VarRef}
\PY{+w}{        }\PY{o}{|}\PY{+w}{ }\PY{l+s}{\PYZsq{}\PYZsh{}\PYZsq{}}\PY{+w}{ }\PY{n+no}{ID}\PY{+w}{                        }\PY{o}{\PYZsh{}}\PY{+w}{ }\PY{n+no}{SymbolLit}
\PY{+w}{        }\PY{o}{|}\PY{+w}{ }\PY{l+s}{\PYZsq{}[\PYZsq{}}\PY{+w}{ }\PY{o}{(}\PY{n+nv}{decl}\PY{o}{*}\PY{+w}{ }\PY{l+s}{\PYZsq{}|\PYZsq{}}\PY{o}{)}\PY{o}{?}\PY{+w}{ }\PY{n+nv}{sequence}\PY{+w}{ }\PY{l+s}{\PYZsq{}]\PYZsq{}}\PY{+w}{ }\PY{o}{\PYZsh{}}\PY{+w}{ }\PY{n+no}{Block}
\PY{+w}{        }\PY{o}{|}\PY{+w}{ }\PY{l+s}{\PYZsq{}(\PYZsq{}}\PY{+w}{ }\PY{n+nv}{expr}\PY{+w}{ }\PY{l+s}{\PYZsq{})\PYZsq{}}\PY{+w}{                  }\PY{o}{\PYZsh{}}\PY{+w}{ }\PY{n+no}{ParenExpr}
\PY{+w}{        }\PY{o}{|}\PY{+w}{ }\PY{l+s}{\PYZsq{}|\PYZsq{}}\PY{+w}{ }\PY{n+nv}{decl}\PY{o}{+}\PY{+w}{ }\PY{l+s}{\PYZsq{}|\PYZsq{}}\PY{+w}{                 }\PY{o}{\PYZsh{}}\PY{+w}{ }\PY{n+no}{DeclExpr}
\PY{+w}{        }\PY{p}{;}

\PY{n+nl}{decl}\PY{+w}{ }\PY{p}{:}\PY{+w}{ }\PY{n+no}{ID}\PY{p}{;}

\PY{c}{//MAYBE: Add return statement (Currently last expression)}
\PY{c}{//TODO: Array syntax (\PYZsq{}\PYZob{}\PYZcb{}\PYZsq{})}

\PY{n+nl}{TRUE}\PY{+w}{ }\PY{p}{:}\PY{+w}{ }\PY{l+s}{\PYZsq{}true\PYZsq{}}\PY{p}{;}
\PY{n+nl}{FALSE}\PY{+w}{ }\PY{p}{:}\PY{+w}{ }\PY{l+s}{\PYZsq{}false\PYZsq{}}\PY{p}{;}
\PY{n+nl}{NIL}\PY{+w}{ }\PY{p}{:}\PY{+w}{ }\PY{l+s}{\PYZsq{}nil\PYZsq{}}\PY{p}{;}

\PY{n+nl}{EXTENDS}\PY{+w}{ }\PY{p}{:}\PY{+w}{ }\PY{l+s}{\PYZsq{}extends\PYZsq{}}\PY{p}{;}

\PY{n+nl}{ID}\PY{p}{:}\PY{+w}{ }\PY{p}{[}\PY{x}{A\PYZhy{}Za\PYZhy{}z}\PY{p}{]}\PY{p}{[}\PY{x}{a\PYZhy{}zA\PYZhy{}Z0\PYZhy{}9\PYZus{}\PYZbs{}\PYZbs{}}\PY{p}{]}\PY{o}{*}\PY{p}{;}
\PY{n+nl}{INTEGER}\PY{+w}{   }\PY{p}{:}\PY{+w}{ }\PY{l+s}{\PYZsq{}\PYZhy{}\PYZsq{}}\PY{o}{?}\PY{+w}{ }\PY{p}{[}\PY{x}{0\PYZhy{}9}\PY{p}{]}\PY{o}{+}\PY{p}{;}
\PY{n+nl}{STRING}\PY{+w}{    }\PY{p}{:}\PY{+w}{ }\PY{l+s}{\PYZsq{}\PYZdq{}\PYZsq{}}\PY{+w}{ }\PY{o}{(}\PY{o}{.}\PY{o}{*}\PY{o}{?}\PY{o}{)}\PY{+w}{ }\PY{l+s}{\PYZsq{}\PYZdq{}\PYZsq{}}\PY{p}{;}

\PY{n+nl}{OPERATOR}\PY{+w}{  }\PY{p}{:}\PY{+w}{ }\PY{o}{(}\PY{l+s}{\PYZsq{}+\PYZsq{}}\PY{+w}{ }\PY{o}{|}\PY{+w}{ }\PY{l+s}{\PYZsq{}\PYZhy{}\PYZsq{}}\PY{+w}{ }\PY{o}{|}\PY{+w}{ }\PY{l+s}{\PYZsq{}*\PYZsq{}}\PY{+w}{ }\PY{o}{|}\PY{+w}{ }\PY{l+s}{\PYZsq{}/\PYZsq{}}\PY{+w}{ }\PY{o}{|}\PY{+w}{ }\PY{l+s}{\PYZsq{}=\PYZsq{}}\PY{+w}{ }\PY{o}{|}\PY{+w}{ }\PY{l+s}{\PYZsq{}!\PYZsq{}}\PY{+w}{ }\PY{o}{|}\PY{+w}{ }\PY{l+s}{\PYZsq{},\PYZsq{}}\PY{+w}{ }\PY{o}{|}\PY{+w}{ }\PY{l+s}{\PYZsq{}\PYZlt{}\PYZsq{}}\PY{+w}{ }\PY{o}{|}\PY{+w}{ }\PY{l+s}{\PYZsq{}\PYZgt{}\PYZsq{}}\PY{+w}{ }\PY{o}{)}\PY{o}{+}\PY{p}{;}

\PY{n+nl}{COMMENT}\PY{+w}{   }\PY{p}{:}\PY{+w}{ }\PY{l+s}{\PYZsq{}/*\PYZsq{}}\PY{+w}{ }\PY{o}{(}\PY{o}{.}\PY{o}{)}\PY{o}{*}\PY{o}{?}\PY{+w}{ }\PY{l+s}{\PYZsq{}*/\PYZsq{}}\PY{+w}{ }\PY{o}{\PYZhy{}\PYZgt{}}\PY{+w}{ }\PY{n+nv}{skip}\PY{p}{;}
\PY{n+nl}{SEPARATOR}\PY{+w}{ }\PY{p}{:}\PY{+w}{ }\PY{p}{[}\PY{x}{ \PYZbs{}t\PYZbs{}r\PYZbs{}n}\PY{p}{]}\PY{+w}{ }\PY{o}{\PYZhy{}\PYZgt{}}\PY{+w}{ }\PY{n+nv}{skip}\PY{p}{;}
\end{Verbatim}


\section{ANTLR tree visitor}

This tree visitor converts an ANTLR parse tree into our intermediate representation, the IR tree.

\begin{Verbatim}[commandchars=\\\{\}]
\PY{k+kn}{package} \PY{n+nn}{org.twnc.irtree}\PY{o}{;}

\PY{k+kn}{import} \PY{n+nn}{java.util.ArrayList}\PY{o}{;}
\PY{k+kn}{import} \PY{n+nn}{java.util.Arrays}\PY{o}{;}
\PY{k+kn}{import} \PY{n+nn}{java.util.List}\PY{o}{;}

\PY{k+kn}{import} \PY{n+nn}{org.antlr.v4.runtime.ParserRuleContext}\PY{o}{;}
\PY{k+kn}{import} \PY{n+nn}{org.antlr.v4.runtime.Token}\PY{o}{;}
\PY{k+kn}{import} \PY{n+nn}{org.antlr.v4.runtime.tree.ParseTree}\PY{o}{;}
\PY{k+kn}{import} \PY{n+nn}{org.antlr.v4.runtime.tree.TerminalNode}\PY{o}{;}
\PY{k+kn}{import} \PY{n+nn}{org.twnc.BabbleBaseVisitor}\PY{o}{;}
\PY{k+kn}{import} \PY{n+nn}{org.twnc.BabbleParser.DeclExprContext}\PY{o}{;}
\PY{k+kn}{import} \PY{n+nn}{org.twnc.BabbleParser.*}\PY{o}{;}
\PY{k+kn}{import} \PY{n+nn}{org.twnc.irtree.nodes.*}\PY{o}{;}
\PY{k+kn}{import} \PY{n+nn}{org.twnc.irtree.nodes.LiteralNode.Type}\PY{o}{;}

\PY{k+kd}{public} \PY{k+kd}{class} \PY{n+nc}{ASTGenerator} \PY{k+kd}{extends} \PY{n}{BabbleBaseVisitor}\PY{o}{\PYZlt{}}\PY{n}{Node}\PY{o}{\PYZgt{}} \PY{o}{\PYZob{}}

    \PY{n+nd}{@Override}
    \PY{k+kd}{public} \PY{n}{Node} \PY{n+nf}{visitTrueLit}\PY{o}{(}\PY{n}{TrueLitContext} \PY{n}{ctx}\PY{o}{)} \PY{o}{\PYZob{}}
        \PY{k}{return} \PY{n}{VarRefNode}\PY{o}{.}\PY{n+na}{newTrue}\PY{o}{(}\PY{o}{)}\PY{o}{;}
    \PY{o}{\PYZcb{}}

    \PY{n+nd}{@Override}
    \PY{k+kd}{public} \PY{n}{Node} \PY{n+nf}{visitNilLit}\PY{o}{(}\PY{n}{NilLitContext} \PY{n}{ctx}\PY{o}{)} \PY{o}{\PYZob{}}
        \PY{k}{return} \PY{n}{VarRefNode}\PY{o}{.}\PY{n+na}{newNil}\PY{o}{(}\PY{o}{)}\PY{o}{;}
    \PY{o}{\PYZcb{}}

    \PY{n+nd}{@Override}
    \PY{k+kd}{public} \PY{n}{Node} \PY{n+nf}{visitSymbolLit}\PY{o}{(}\PY{n}{SymbolLitContext} \PY{n}{ctx}\PY{o}{)} \PY{o}{\PYZob{}}
        \PY{k}{return} \PY{k}{new} \PY{n}{LiteralNode}\PY{o}{(}\PY{n}{Type}\PY{o}{.}\PY{n+na}{SYMBOL}\PY{o}{,} \PY{n}{ctx}\PY{o}{.}\PY{n+na}{ID}\PY{o}{(}\PY{o}{)}\PY{o}{.}\PY{n+na}{getText}\PY{o}{(}\PY{o}{)}\PY{o}{)}\PY{o}{;}
    \PY{o}{\PYZcb{}}

    \PY{n+nd}{@Override}
    \PY{k+kd}{public} \PY{n}{Node} \PY{n+nf}{visitProgram}\PY{o}{(}\PY{n}{ProgramContext} \PY{n}{ctx}\PY{o}{)} \PY{o}{\PYZob{}}
        \PY{n}{List}\PY{o}{\PYZlt{}}\PY{n}{ClazzNode}\PY{o}{\PYZgt{}} \PY{n}{classes} \PY{o}{=} \PY{k}{new} \PY{n}{ArrayList}\PY{o}{\PYZlt{}}\PY{n}{ClazzNode}\PY{o}{\PYZgt{}}\PY{o}{(}\PY{o}{)}\PY{o}{;}
        \PY{k}{for} \PY{o}{(}\PY{n}{ClazzContext} \PY{n}{context} \PY{o}{:} \PY{n}{ctx}\PY{o}{.}\PY{n+na}{clazz}\PY{o}{(}\PY{o}{)}\PY{o}{)} \PY{o}{\PYZob{}}
            \PY{n}{classes}\PY{o}{.}\PY{n+na}{add}\PY{o}{(}\PY{o}{(}\PY{n}{ClazzNode}\PY{o}{)} \PY{n}{visit}\PY{o}{(}\PY{n}{context}\PY{o}{)}\PY{o}{)}\PY{o}{;}
        \PY{o}{\PYZcb{}}

        \PY{k}{return} \PY{k}{new} \PY{n}{ProgramNode}\PY{o}{(}\PY{n}{classes}\PY{o}{)}\PY{o}{;}
    \PY{o}{\PYZcb{}}

    \PY{n+nd}{@Override}
    \PY{k+kd}{public} \PY{n}{Node} \PY{n+nf}{visitClazz}\PY{o}{(}\PY{n}{ClazzContext} \PY{n}{ctx}\PY{o}{)} \PY{o}{\PYZob{}}
        \PY{n}{String} \PY{n}{superclass}\PY{o}{;}
        \PY{n}{String} \PY{n}{clazzName} \PY{o}{=} \PY{n}{ctx}\PY{o}{.}\PY{n+na}{classname}\PY{o}{.}\PY{n+na}{getText}\PY{o}{(}\PY{o}{)}\PY{o}{;}

        \PY{k}{if} \PY{o}{(}\PY{n}{ctx}\PY{o}{.}\PY{n+na}{superclass} \PY{o}{!}\PY{o}{=} \PY{k+kc}{null}\PY{o}{)} \PY{o}{\PYZob{}}
            \PY{n}{superclass} \PY{o}{=} \PY{n}{ctx}\PY{o}{.}\PY{n+na}{superclass}\PY{o}{.}\PY{n+na}{getText}\PY{o}{(}\PY{o}{)}\PY{o}{.}\PY{n+na}{replace}\PY{o}{(}\PY{l+s+sc}{\PYZsq{}\PYZbs{}\PYZbs{}\PYZsq{}}\PY{o}{,} \PY{l+s+sc}{\PYZsq{}/\PYZsq{}}\PY{o}{)}\PY{o}{;}
        \PY{o}{\PYZcb{}} \PY{k}{else} \PY{o}{\PYZob{}}
            \PY{n}{superclass} \PY{o}{=} \PY{l+s}{\PYZdq{}org/twnc/runtime/BObject\PYZdq{}}\PY{o}{;}
        \PY{o}{\PYZcb{}}

        \PY{n}{List}\PY{o}{\PYZlt{}}\PY{n}{MethodNode}\PY{o}{\PYZgt{}} \PY{n}{methods} \PY{o}{=} \PY{k}{new} \PY{n}{ArrayList}\PY{o}{\PYZlt{}}\PY{o}{\PYZgt{}}\PY{o}{(}\PY{o}{)}\PY{o}{;}
        \PY{k}{for} \PY{o}{(}\PY{n}{MthdContext} \PY{n}{m} \PY{o}{:} \PY{n}{ctx}\PY{o}{.}\PY{n+na}{mthd}\PY{o}{(}\PY{o}{)}\PY{o}{)} \PY{o}{\PYZob{}}
            \PY{n}{methods}\PY{o}{.}\PY{n+na}{add}\PY{o}{(}\PY{o}{(}\PY{n}{MethodNode}\PY{o}{)} \PY{n}{visit}\PY{o}{(}\PY{n}{m}\PY{o}{)}\PY{o}{)}\PY{o}{;}
        \PY{o}{\PYZcb{}}
        
        \PY{k}{return} \PY{k}{new} \PY{n}{ClazzNode}\PY{o}{(}\PY{n}{clazzName}\PY{o}{,} \PY{n}{superclass}\PY{o}{,} \PY{n}{methods}\PY{o}{)}\PY{o}{;}
    \PY{o}{\PYZcb{}}

    \PY{n+nd}{@Override}
    \PY{k+kd}{public} \PY{n}{Node} \PY{n+nf}{visitBlock}\PY{o}{(}\PY{n}{BlockContext} \PY{n}{ctx}\PY{o}{)} \PY{o}{\PYZob{}}
        \PY{n}{List}\PY{o}{\PYZlt{}}\PY{n}{VarRefNode}\PY{o}{\PYZgt{}} \PY{n}{arguments} \PY{o}{=} \PY{k}{new} \PY{n}{ArrayList}\PY{o}{\PYZlt{}}\PY{o}{\PYZgt{}}\PY{o}{(}\PY{o}{)}\PY{o}{;}
        \PY{k}{for} \PY{o}{(}\PY{n}{DeclContext} \PY{n}{node} \PY{o}{:} \PY{n}{ctx}\PY{o}{.}\PY{n+na}{decl}\PY{o}{(}\PY{o}{)}\PY{o}{)} \PY{o}{\PYZob{}}
            \PY{n}{arguments}\PY{o}{.}\PY{n+na}{add}\PY{o}{(}\PY{k}{new} \PY{n}{VarRefNode}\PY{o}{(}\PY{n}{node}\PY{o}{.}\PY{n+na}{getText}\PY{o}{(}\PY{o}{)}\PY{o}{)}\PY{o}{)}\PY{o}{;}
        \PY{o}{\PYZcb{}}
        \PY{n}{SequenceNode} \PY{n}{sequence} \PY{o}{=} \PY{o}{(}\PY{n}{SequenceNode}\PY{o}{)} \PY{n}{visit}\PY{o}{(}\PY{n}{ctx}\PY{o}{.}\PY{n+na}{sequence}\PY{o}{(}\PY{o}{)}\PY{o}{)}\PY{o}{;}
        \PY{k}{return} \PY{k}{new} \PY{n}{BlockNode}\PY{o}{(}\PY{n}{sequence}\PY{o}{,} \PY{n}{arguments}\PY{o}{)}\PY{o}{;}
    \PY{o}{\PYZcb{}}

    \PY{n+nd}{@Override}
    \PY{k+kd}{public} \PY{n}{Node} \PY{n+nf}{visitFalseLit}\PY{o}{(}\PY{n}{FalseLitContext} \PY{n}{ctx}\PY{o}{)} \PY{o}{\PYZob{}}
        \PY{k}{return} \PY{n}{VarRefNode}\PY{o}{.}\PY{n+na}{newFalse}\PY{o}{(}\PY{o}{)}\PY{o}{;}
    \PY{o}{\PYZcb{}}

    \PY{n+nd}{@Override}
    \PY{k+kd}{public} \PY{n}{Node} \PY{n+nf}{visitAssignment}\PY{o}{(}\PY{n}{AssignmentContext} \PY{n}{ctx}\PY{o}{)} \PY{o}{\PYZob{}}
        \PY{n}{VarRefNode} \PY{n}{variable} \PY{o}{=} \PY{k}{new} \PY{n}{VarRefNode}\PY{o}{(}\PY{n}{ctx}\PY{o}{.}\PY{n+na}{ID}\PY{o}{(}\PY{o}{)}\PY{o}{.}\PY{n+na}{getText}\PY{o}{(}\PY{o}{)}\PY{o}{)}\PY{o}{;}
        \PY{n}{ExprNode} \PY{n}{expression} \PY{o}{=} \PY{o}{(}\PY{n}{ExprNode}\PY{o}{)} \PY{n}{visit}\PY{o}{(}\PY{n}{ctx}\PY{o}{.}\PY{n+na}{expr}\PY{o}{(}\PY{o}{)}\PY{o}{)}\PY{o}{;}
        \PY{k}{return} \PY{k}{new} \PY{n}{AssignNode}\PY{o}{(}\PY{n}{variable}\PY{o}{,} \PY{n}{expression}\PY{o}{)}\PY{o}{;}
    \PY{o}{\PYZcb{}}

    \PY{n+nd}{@Override}
    \PY{k+kd}{public} \PY{n}{Node} \PY{n+nf}{visitSequence}\PY{o}{(}\PY{n}{SequenceContext} \PY{n}{ctx}\PY{o}{)} \PY{o}{\PYZob{}}
        \PY{k}{return} \PY{k}{new} \PY{n}{SequenceNode}\PY{o}{(}\PY{n}{visitExprArguments}\PY{o}{(}\PY{n}{ctx}\PY{o}{.}\PY{n+na}{expr}\PY{o}{(}\PY{o}{)}\PY{o}{)}\PY{o}{)}\PY{o}{;}
    \PY{o}{\PYZcb{}}

    \PY{n+nd}{@Override}
    \PY{k+kd}{public} \PY{n}{Node} \PY{n+nf}{visitGlobalKeywordSend}\PY{o}{(}\PY{n}{GlobalKeywordSendContext} \PY{n}{ctx}\PY{o}{)} \PY{o}{\PYZob{}}
        \PY{n}{String} \PY{n}{selector} \PY{o}{=} \PY{n}{buildSelector}\PY{o}{(}\PY{n}{ctx}\PY{o}{.}\PY{n+na}{ID}\PY{o}{(}\PY{o}{)}\PY{o}{)}\PY{o}{;}
        \PY{n}{List}\PY{o}{\PYZlt{}}\PY{n}{ExprNode}\PY{o}{\PYZgt{}} \PY{n}{arguments} \PY{o}{=} \PY{n}{visitExprArguments}\PY{o}{(}\PY{n}{ctx}\PY{o}{.}\PY{n+na}{subexpr}\PY{o}{(}\PY{o}{)}\PY{o}{)}\PY{o}{;}
        \PY{k}{return} \PY{k}{new} \PY{n}{SendNode}\PY{o}{(}\PY{n}{selector}\PY{o}{,} \PY{n}{arguments}\PY{o}{)}\PY{o}{;}
    \PY{o}{\PYZcb{}}

    \PY{n+nd}{@Override}
    \PY{k+kd}{public} \PY{n}{Node} \PY{n+nf}{visitKeywordSend}\PY{o}{(}\PY{n}{KeywordSendContext} \PY{n}{ctx}\PY{o}{)} \PY{o}{\PYZob{}}
        \PY{n}{ExprNode} \PY{n}{expression} \PY{o}{=} \PY{o}{(}\PY{n}{ExprNode}\PY{o}{)} \PY{n}{visit}\PY{o}{(}\PY{n}{ctx}\PY{o}{.}\PY{n+na}{expr}\PY{o}{(}\PY{o}{)}\PY{o}{)}\PY{o}{;}
        \PY{n}{String} \PY{n}{selector} \PY{o}{=} \PY{n}{buildSelector}\PY{o}{(}\PY{n}{ctx}\PY{o}{.}\PY{n+na}{ID}\PY{o}{(}\PY{o}{)}\PY{o}{)}\PY{o}{;}
        \PY{n}{List}\PY{o}{\PYZlt{}}\PY{n}{ExprNode}\PY{o}{\PYZgt{}} \PY{n}{arguments} \PY{o}{=} \PY{n}{visitExprArguments}\PY{o}{(}\PY{n}{ctx}\PY{o}{.}\PY{n+na}{subexpr}\PY{o}{(}\PY{o}{)}\PY{o}{)}\PY{o}{;}
        \PY{k}{return} \PY{k}{new} \PY{n}{SendNode}\PY{o}{(}\PY{n}{expression}\PY{o}{,} \PY{n}{selector}\PY{o}{,} \PY{n}{arguments}\PY{o}{)}\PY{o}{;}
    \PY{o}{\PYZcb{}}

    \PY{n+nd}{@Override}
    \PY{k+kd}{public} \PY{n}{Node} \PY{n+nf}{visitKeywordMethod}\PY{o}{(}\PY{n}{KeywordMethodContext} \PY{n}{ctx}\PY{o}{)} \PY{o}{\PYZob{}}
        \PY{n}{String} \PY{n}{selector} \PY{o}{=} \PY{l+s}{\PYZdq{}\PYZdq{}}\PY{o}{;}
        \PY{n}{List}\PY{o}{\PYZlt{}}\PY{n}{VarRefNode}\PY{o}{\PYZgt{}} \PY{n}{arguments} \PY{o}{=} \PY{k}{new} \PY{n}{ArrayList}\PY{o}{\PYZlt{}}\PY{n}{VarRefNode}\PY{o}{\PYZgt{}}\PY{o}{(}\PY{o}{)}\PY{o}{;}
        \PY{k}{for} \PY{o}{(}\PY{k+kt}{int} \PY{n}{i} \PY{o}{=} \PY{l+m+mi}{0}\PY{o}{;} \PY{n}{i} \PY{o}{\PYZlt{}} \PY{n}{ctx}\PY{o}{.}\PY{n+na}{ID}\PY{o}{(}\PY{o}{)}\PY{o}{.}\PY{n+na}{size}\PY{o}{(}\PY{o}{)}\PY{o}{;} \PY{n}{i} \PY{o}{+}\PY{o}{=} \PY{l+m+mi}{2}\PY{o}{)} \PY{o}{\PYZob{}}
            \PY{n}{selector} \PY{o}{+}\PY{o}{=} \PY{n}{ctx}\PY{o}{.}\PY{n+na}{ID}\PY{o}{(}\PY{n}{i}\PY{o}{)} \PY{o}{+} \PY{l+s}{\PYZdq{}:\PYZdq{}}\PY{o}{;}
            \PY{n}{arguments}\PY{o}{.}\PY{n+na}{add}\PY{o}{(}\PY{k}{new} \PY{n}{VarRefNode}\PY{o}{(}\PY{n}{ctx}\PY{o}{.}\PY{n+na}{ID}\PY{o}{(}\PY{n}{i} \PY{o}{+} \PY{l+m+mi}{1}\PY{o}{)}\PY{o}{.}\PY{n+na}{getText}\PY{o}{(}\PY{o}{)}\PY{o}{)}\PY{o}{)}\PY{o}{;}
        \PY{o}{\PYZcb{}}
        \PY{n}{SequenceNode} \PY{n}{sequence} \PY{o}{=} \PY{o}{(}\PY{n}{SequenceNode}\PY{o}{)} \PY{n}{visit}\PY{o}{(}\PY{n}{ctx}\PY{o}{.}\PY{n+na}{sequence}\PY{o}{(}\PY{o}{)}\PY{o}{)}\PY{o}{;}

        \PY{k}{return} \PY{k}{new} \PY{n}{MethodNode}\PY{o}{(}\PY{n}{selector}\PY{o}{,} \PY{n}{arguments}\PY{o}{,} \PY{n}{sequence}\PY{o}{)}\PY{o}{;}
    \PY{o}{\PYZcb{}}
    
    \PY{n+nd}{@Override}
    \PY{k+kd}{public} \PY{n}{Node} \PY{n+nf}{visitUnaryMethod}\PY{o}{(}\PY{n}{UnaryMethodContext} \PY{n}{ctx}\PY{o}{)} \PY{o}{\PYZob{}}
        \PY{n}{String} \PY{n}{selector} \PY{o}{=} \PY{n}{ctx}\PY{o}{.}\PY{n+na}{ID}\PY{o}{(}\PY{o}{)}\PY{o}{.}\PY{n+na}{getText}\PY{o}{(}\PY{o}{)}\PY{o}{;}
        \PY{n}{SequenceNode} \PY{n}{sequence} \PY{o}{=} \PY{o}{(}\PY{n}{SequenceNode}\PY{o}{)} \PY{n}{visit}\PY{o}{(}\PY{n}{ctx}\PY{o}{.}\PY{n+na}{sequence}\PY{o}{(}\PY{o}{)}\PY{o}{)}\PY{o}{;}

        \PY{k}{return} \PY{k}{new} \PY{n}{MethodNode}\PY{o}{(}\PY{n}{selector}\PY{o}{,} \PY{n}{sequence}\PY{o}{)}\PY{o}{;}
    \PY{o}{\PYZcb{}}

    \PY{n+nd}{@Override}
    \PY{k+kd}{public} \PY{n}{Node} \PY{n+nf}{visitInfixSend}\PY{o}{(}\PY{n}{InfixSendContext} \PY{n}{ctx}\PY{o}{)} \PY{o}{\PYZob{}}
        \PY{n}{ExprNode} \PY{n}{expression} \PY{o}{=} \PY{o}{(}\PY{n}{ExprNode}\PY{o}{)} \PY{n}{visit}\PY{o}{(}\PY{n}{ctx}\PY{o}{.}\PY{n+na}{rcv}\PY{o}{)}\PY{o}{;}
        \PY{n}{String} \PY{n}{selector} \PY{o}{=} \PY{n}{ctx}\PY{o}{.}\PY{n+na}{method}\PY{o}{.}\PY{n+na}{getText}\PY{o}{(}\PY{o}{)} \PY{o}{+} \PY{l+s+sc}{\PYZsq{}:\PYZsq{}}\PY{o}{;}
        \PY{n}{List}\PY{o}{\PYZlt{}}\PY{n}{ExprNode}\PY{o}{\PYZgt{}} \PY{n}{arguments} \PY{o}{=} \PY{n}{visitExprArguments}\PY{o}{(}\PY{n}{ctx}\PY{o}{.}\PY{n+na}{arg}\PY{o}{)}\PY{o}{;}
        \PY{k}{return} \PY{k}{new} \PY{n}{SendNode}\PY{o}{(}\PY{n}{expression}\PY{o}{,} \PY{n}{selector}\PY{o}{,} \PY{n}{arguments}\PY{o}{)}\PY{o}{;}
    \PY{o}{\PYZcb{}}

    \PY{n+nd}{@Override}
    \PY{k+kd}{public} \PY{n}{Node} \PY{n+nf}{visitUnarySend}\PY{o}{(}\PY{n}{UnarySendContext} \PY{n}{ctx}\PY{o}{)} \PY{o}{\PYZob{}}
        \PY{n}{ExprNode} \PY{n}{expression} \PY{o}{=} \PY{o}{(}\PY{n}{ExprNode}\PY{o}{)} \PY{n}{visit}\PY{o}{(}\PY{n}{ctx}\PY{o}{.}\PY{n+na}{expr}\PY{o}{(}\PY{o}{)}\PY{o}{)}\PY{o}{;}
        \PY{n}{String} \PY{n}{selector} \PY{o}{=} \PY{n}{ctx}\PY{o}{.}\PY{n+na}{ID}\PY{o}{(}\PY{o}{)}\PY{o}{.}\PY{n+na}{getText}\PY{o}{(}\PY{o}{)}\PY{o}{;}
        \PY{k}{return} \PY{k}{new} \PY{n}{SendNode}\PY{o}{(}\PY{n}{expression}\PY{o}{,} \PY{n}{selector}\PY{o}{)}\PY{o}{;}
    \PY{o}{\PYZcb{}}

    \PY{n+nd}{@Override}
    \PY{k+kd}{public} \PY{n}{Node} \PY{n+nf}{visitStrLit}\PY{o}{(}\PY{n}{StrLitContext} \PY{n}{ctx}\PY{o}{)} \PY{o}{\PYZob{}}
        \PY{n}{String} \PY{n}{quoted} \PY{o}{=} \PY{n}{ctx}\PY{o}{.}\PY{n+na}{string}\PY{o}{.}\PY{n+na}{getText}\PY{o}{(}\PY{o}{)}\PY{o}{;}
        \PY{n}{String} \PY{n}{unquoted} \PY{o}{=} \PY{n}{quoted}\PY{o}{.}\PY{n+na}{substring}\PY{o}{(}\PY{l+m+mi}{1}\PY{o}{,} \PY{n}{quoted}\PY{o}{.}\PY{n+na}{length}\PY{o}{(}\PY{o}{)} \PY{o}{\PYZhy{}} \PY{l+m+mi}{1}\PY{o}{)}\PY{o}{;}
        \PY{k}{return} \PY{k}{new} \PY{n}{LiteralNode}\PY{o}{(}\PY{n}{Type}\PY{o}{.}\PY{n+na}{STRING}\PY{o}{,} \PY{n}{unquoted}\PY{o}{)}\PY{o}{;}
    \PY{o}{\PYZcb{}}

    \PY{n+nd}{@Override}
    \PY{k+kd}{public} \PY{n}{Node} \PY{n+nf}{visitVarRef}\PY{o}{(}\PY{n}{VarRefContext} \PY{n}{ctx}\PY{o}{)} \PY{o}{\PYZob{}}
        \PY{k}{return} \PY{k}{new} \PY{n}{VarRefNode}\PY{o}{(}\PY{n}{ctx}\PY{o}{.}\PY{n+na}{getText}\PY{o}{(}\PY{o}{)}\PY{o}{)}\PY{o}{;}
    \PY{o}{\PYZcb{}}

    \PY{n+nd}{@Override}
    \PY{k+kd}{public} \PY{n}{Node} \PY{n+nf}{visitIntLit}\PY{o}{(}\PY{n}{IntLitContext} \PY{n}{ctx}\PY{o}{)} \PY{o}{\PYZob{}}
        \PY{k}{return} \PY{k}{new} \PY{n}{LiteralNode}\PY{o}{(}\PY{n}{Type}\PY{o}{.}\PY{n+na}{INTEGER}\PY{o}{,} \PY{n}{ctx}\PY{o}{.}\PY{n+na}{getText}\PY{o}{(}\PY{o}{)}\PY{o}{)}\PY{o}{;}
    \PY{o}{\PYZcb{}}

    \PY{n+nd}{@Override}
    \PY{k+kd}{public} \PY{n}{Node} \PY{n+nf}{visitParenExpr}\PY{o}{(}\PY{n}{ParenExprContext} \PY{n}{ctx}\PY{o}{)} \PY{o}{\PYZob{}}
        \PY{k}{return} \PY{n}{visit}\PY{o}{(}\PY{n}{ctx}\PY{o}{.}\PY{n+na}{expr}\PY{o}{(}\PY{o}{)}\PY{o}{)}\PY{o}{;}
    \PY{o}{\PYZcb{}}

    \PY{n+nd}{@Override}
    \PY{k+kd}{public} \PY{n}{Node} \PY{n+nf}{visitLoneExpr}\PY{o}{(}\PY{n}{LoneExprContext} \PY{n}{ctx}\PY{o}{)} \PY{o}{\PYZob{}}
        \PY{k}{return} \PY{n}{visit}\PY{o}{(}\PY{n}{ctx}\PY{o}{.}\PY{n+na}{subexpr}\PY{o}{(}\PY{o}{)}\PY{o}{)}\PY{o}{;}
    \PY{o}{\PYZcb{}}    
    
    \PY{n+nd}{@Override}
    \PY{k+kd}{public} \PY{n}{Node} \PY{n+nf}{visitDecl}\PY{o}{(}\PY{n}{DeclContext} \PY{n}{ctx}\PY{o}{)} \PY{o}{\PYZob{}}
        \PY{k}{return} \PY{k}{new} \PY{n}{VarDeclNode}\PY{o}{(}\PY{n}{ctx}\PY{o}{.}\PY{n+na}{getText}\PY{o}{(}\PY{o}{)}\PY{o}{)}\PY{o}{;}
    \PY{o}{\PYZcb{}}
    
    \PY{n+nd}{@Override}
    \PY{k+kd}{public} \PY{n}{Node} \PY{n+nf}{visitDeclExpr}\PY{o}{(}\PY{n}{DeclExprContext} \PY{n}{ctx}\PY{o}{)} \PY{o}{\PYZob{}}
        \PY{k}{return} \PY{k}{new} \PY{n}{DeclExprNode}\PY{o}{(}\PY{n}{visitDecls}\PY{o}{(}\PY{n}{ctx}\PY{o}{.}\PY{n+na}{decl}\PY{o}{(}\PY{o}{)}\PY{o}{)}\PY{o}{)}\PY{o}{;}
        
    \PY{o}{\PYZcb{}}

    \PY{n+nd}{@Override}
    \PY{k+kd}{public} \PY{n}{Node} \PY{n+nf}{visit}\PY{o}{(}\PY{n}{ParseTree} \PY{n}{tree}\PY{o}{)} \PY{o}{\PYZob{}}
        \PY{n}{Node} \PY{n}{n} \PY{o}{=} \PY{k+kd}{super}\PY{o}{.}\PY{n+na}{visit}\PY{o}{(}\PY{n}{tree}\PY{o}{)}\PY{o}{;}
        \PY{k}{if} \PY{o}{(}\PY{n}{tree} \PY{k}{instanceof} \PY{n}{ParserRuleContext}\PY{o}{)} \PY{o}{\PYZob{}}
            \PY{n}{Token} \PY{n}{tok} \PY{o}{=} \PY{o}{(}\PY{o}{(}\PY{n}{ParserRuleContext}\PY{o}{)}\PY{n}{tree}\PY{o}{)}\PY{o}{.}\PY{n+na}{start}\PY{o}{;}
            \PY{n}{n}\PY{o}{.}\PY{n+na}{setLine}\PY{o}{(}\PY{n}{tok}\PY{o}{.}\PY{n+na}{getLine}\PY{o}{(}\PY{o}{)}\PY{o}{)}\PY{o}{;}
            \PY{n}{n}\PY{o}{.}\PY{n+na}{setLineOffset}\PY{o}{(}\PY{n}{tok}\PY{o}{.}\PY{n+na}{getLine}\PY{o}{(}\PY{o}{)}\PY{o}{,} \PY{n}{tok}\PY{o}{.}\PY{n+na}{getCharPositionInLine}\PY{o}{(}\PY{o}{)}\PY{o}{)}\PY{o}{;}
        \PY{o}{\PYZcb{}}

        \PY{k}{return} \PY{n}{n}\PY{o}{;}
    \PY{o}{\PYZcb{}}

    \PY{k+kd}{public} \PY{n}{List}\PY{o}{\PYZlt{}}\PY{n}{ExprNode}\PY{o}{\PYZgt{}} \PY{n+nf}{visitExprArguments}\PY{o}{(}\PY{n}{ParserRuleContext} \PY{n}{argument}\PY{o}{)} \PY{o}{\PYZob{}}
        \PY{k}{return} \PY{n}{visitExprArguments}\PY{o}{(}\PY{n}{Arrays}\PY{o}{.}\PY{n+na}{asList}\PY{o}{(}\PY{n}{argument}\PY{o}{)}\PY{o}{)}\PY{o}{;}
    \PY{o}{\PYZcb{}}

    \PY{k+kd}{public} \PY{n}{List}\PY{o}{\PYZlt{}}\PY{n}{ExprNode}\PY{o}{\PYZgt{}} \PY{n+nf}{visitExprArguments}\PY{o}{(}\PY{n}{List}\PY{o}{\PYZlt{}}\PY{o}{?} \PY{k+kd}{extends} \PY{n}{ParserRuleContext}\PY{o}{\PYZgt{}} \PY{n}{arguments}\PY{o}{)} \PY{o}{\PYZob{}}
        \PY{n}{List}\PY{o}{\PYZlt{}}\PY{n}{ExprNode}\PY{o}{\PYZgt{}} \PY{n}{output} \PY{o}{=} \PY{k}{new} \PY{n}{ArrayList}\PY{o}{\PYZlt{}}\PY{o}{\PYZgt{}}\PY{o}{(}\PY{o}{)}\PY{o}{;}
        \PY{k}{for} \PY{o}{(}\PY{k+kt}{int} \PY{n}{i} \PY{o}{=} \PY{l+m+mi}{0}\PY{o}{;} \PY{n}{i} \PY{o}{\PYZlt{}} \PY{n}{arguments}\PY{o}{.}\PY{n+na}{size}\PY{o}{(}\PY{o}{)}\PY{o}{;} \PY{n}{i} \PY{o}{+}\PY{o}{=} \PY{l+m+mi}{1}\PY{o}{)} \PY{o}{\PYZob{}}
            \PY{n}{output}\PY{o}{.}\PY{n+na}{add}\PY{o}{(}\PY{o}{(}\PY{n}{ExprNode}\PY{o}{)} \PY{n}{visit}\PY{o}{(}\PY{n}{arguments}\PY{o}{.}\PY{n+na}{get}\PY{o}{(}\PY{n}{i}\PY{o}{)}\PY{o}{)}\PY{o}{)}\PY{o}{;}
        \PY{o}{\PYZcb{}}
        \PY{k}{return} \PY{n}{output}\PY{o}{;}
    \PY{o}{\PYZcb{}}
    
    \PY{k+kd}{public} \PY{n}{List}\PY{o}{\PYZlt{}}\PY{n}{VarDeclNode}\PY{o}{\PYZgt{}} \PY{n+nf}{visitDecls}\PY{o}{(}\PY{n}{List}\PY{o}{\PYZlt{}}\PY{o}{?} \PY{k+kd}{extends} \PY{n}{ParserRuleContext}\PY{o}{\PYZgt{}} \PY{n}{decls}\PY{o}{)} \PY{o}{\PYZob{}}
        \PY{n}{List}\PY{o}{\PYZlt{}}\PY{n}{VarDeclNode}\PY{o}{\PYZgt{}} \PY{n}{output} \PY{o}{=} \PY{k}{new} \PY{n}{ArrayList}\PY{o}{\PYZlt{}}\PY{o}{\PYZgt{}}\PY{o}{(}\PY{o}{)}\PY{o}{;}
        \PY{k}{for} \PY{o}{(}\PY{k+kt}{int} \PY{n}{i} \PY{o}{=} \PY{l+m+mi}{0}\PY{o}{;} \PY{n}{i} \PY{o}{\PYZlt{}} \PY{n}{decls}\PY{o}{.}\PY{n+na}{size}\PY{o}{(}\PY{o}{)}\PY{o}{;} \PY{n}{i} \PY{o}{+}\PY{o}{=} \PY{l+m+mi}{1}\PY{o}{)} \PY{o}{\PYZob{}}
            \PY{n}{output}\PY{o}{.}\PY{n+na}{add}\PY{o}{(}\PY{o}{(}\PY{n}{VarDeclNode}\PY{o}{)} \PY{n}{visit}\PY{o}{(}\PY{n}{decls}\PY{o}{.}\PY{n+na}{get}\PY{o}{(}\PY{n}{i}\PY{o}{)}\PY{o}{)}\PY{o}{)}\PY{o}{;}
        \PY{o}{\PYZcb{}}
        \PY{k}{return} \PY{n}{output}\PY{o}{;}
    \PY{o}{\PYZcb{}}
    
    \PY{k+kd}{public} \PY{n}{String} \PY{n+nf}{buildSelector}\PY{o}{(}\PY{n}{List}\PY{o}{\PYZlt{}}\PY{n}{TerminalNode}\PY{o}{\PYZgt{}} \PY{n}{list}\PY{o}{)} \PY{o}{\PYZob{}}
        \PY{n}{String} \PY{n}{selector} \PY{o}{=} \PY{l+s}{\PYZdq{}\PYZdq{}}\PY{o}{;}
        \PY{k}{for} \PY{o}{(}\PY{k+kt}{int} \PY{n}{i} \PY{o}{=} \PY{l+m+mi}{0}\PY{o}{;} \PY{n}{i} \PY{o}{\PYZlt{}} \PY{n}{list}\PY{o}{.}\PY{n+na}{size}\PY{o}{(}\PY{o}{)}\PY{o}{;} \PY{n}{i} \PY{o}{+}\PY{o}{=} \PY{l+m+mi}{1}\PY{o}{)} \PY{o}{\PYZob{}}
            \PY{n}{selector} \PY{o}{+}\PY{o}{=} \PY{n}{list}\PY{o}{.}\PY{n+na}{get}\PY{o}{(}\PY{n}{i}\PY{o}{)}\PY{o}{.}\PY{n+na}{toString}\PY{o}{(}\PY{o}{)} \PY{o}{+} \PY{l+s}{\PYZdq{}:\PYZdq{}}\PY{o}{;}
        \PY{o}{\PYZcb{}}
        \PY{k}{return} \PY{n}{selector}\PY{o}{;}
    \PY{o}{\PYZcb{}}
\PY{o}{\PYZcb{}}
\end{Verbatim}


\end{appendices}

\end{document}
